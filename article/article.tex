\documentclass[a4paper,11pt,twoside,french]{article}
\usepackage[OT1]{fontenc} \usepackage[utf8]{inputenc}
\usepackage{babel}
\usepackage[a4paper,tmargin=2.5truecm,bmargin=3truecm,outer=2.5truecm,inner=2.5truecm,twoside,verbose=false %,showframe
]{geometry}

\usepackage{amsmath,amssymb,amsthm}
\usepackage{multicol}

\begin{document}

    \begin{multicols}{2}

\vfill\null
\columnbreak
At Tevatron, $t\bar{t}5$ production was initiated mainly by $q\bar{q}$ annihilation while at the LHC, $gg$ fusion is dominant. We compare the $f(t)$ amplitude, between samples generated at the same center-of-mass energy $\sqrt{s} = 1.96$ TeV for D$\emptyset$ and CMS. We find similar amplitudes between D$\emptyset$ and CMS at $\sqrt{s} = 1.96$ TeV for the benchmarks $c_{XX} = -c_{YY} \neq 0$ and $c_{XY} = c_{YX} \neq 0$. However, at the same energy and production mechanism, the LHC position induces worst expected sensitivity to $c_{XZ} = c_{ZX} \neq 0$ and $c_{YZ} = c_{ZY} \neq 0$ benchmarks. We scanned the latitude and azimuth of poential experiments on earth and foiund that both ATLAS or CMS sit in a dip for the projected sensitivity on those SME coefficients.

    \end{multicols}


\end{document}