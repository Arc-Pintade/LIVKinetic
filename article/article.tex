\documentclass[a4paper,11pt,twoside,french]{article}
\usepackage[OT1]{fontenc} \usepackage[utf8]{inputenc}
\usepackage{babel}
\usepackage[a4paper,tmargin=2.5truecm,bmargin=3truecm,outer=2.5truecm,inner=2.5truecm,twoside,verbose=false %,showframe
]{geometry}

\usepackage{amsmath,amssymb,amsthm}
\usepackage{multicol}
\usepackage{enumitem}
\usepackage{graphicx}

\begin{document}


%%%%%%%%%%%% Première partie avec plots

Amplitudes of the $f(t)$ functions, at selected center-of-mass energies, is shown on Figure~\ref{ampl}. Greatest modulations of $t\bar{t}$ cross sections are expected in the benchmark scenarios $c_{XY} = c_{YX}$ and $c_{XX} = -c_{YY}$. The sensitivity is higher when probing components in the ecliptic plane.

At Tevatron, $t\bar{t}$ production was initiated mainly by $q\bar{q}$ annihilation while at the LHC, $gg$ fusion is dominant. We compare the $f(t)$ amplitude, between samples generated at the same center-of-mass energy $\sqrt{s} = 1.96$ TeV for D$\emptyset$ and CMS. We find similar amplitudes between D$\emptyset$ and CMS at $\sqrt{s} = 1.96$ TeV for the benchmarks $c_{XX} = -c_{YY} \neq 0$ and $c_{XY} = c_{YX} \neq 0$. However, at the same energy and production mechanism, the LHC position induces worst expected sensitivity to $c_{XZ} = c_{ZX} \neq 0$ and $c_{YZ} = c_{ZY} \neq 0$ benchmarks. We scanned the latitude and azimuth of poential experiments on earth and foiund that both ATLAS or CMS sit in a dip for the projected sensitivity on those SME coefficients.

    \begin{figure}[h!]
        \begin{center}
            \label{ampl:1}
            \includegraphics[scale=0.4]{amplEnergy_{XX}.eps}
            \includegraphics[scale=0.4]{amplEnergy_{XY}.eps}
            \includegraphics[scale=0.4]{amplEnergy_{XZ}.eps}
            \includegraphics[scale=0.4]{amplEnergy_{YZ}.eps}
            \caption{Amplitude for experiments D$\emptyset$, CMS (LHC), CMS (HL-LHC), CMS (HE-LHC) and FCC for all $c_{\mu \nu}$ benchmark}
        \end{center}
    \end{figure}

%%%%%%%%%%%% Deuxième partie avec table

We compute the projected precision on the SME coefficients with $\mbox{\textsc{HistFactory}}$ \cite{Cranmer:2012sba}, using the Asimov dataset, for the above mentioned collider and SME coefficient benchmarks. Histograms for LIV signal, SM $t\bar{t}$ production and single top background are provided, with bins of one sidereal hour. Systematic uncertainties are rounded from \cite{Khachatryan:2016kzg}: 2\% is attributed to the luminosity, 4\% on the inclusive measurement of $t\bar{t}$ production, and 2\% on single top production. These projections are shown on Table~\ref{tab:1}.
\begin{table}[h!]
	% table caption is above the table
	\caption{Comparison of $f(t)$ amplitudes in $t\bar{t}$ signature in $p - p$ collisions at CMS position at 1.96, 7, 8, 13, 14, 27 and 100 TeV.}
	\label{tab:1}       % Give a unique label
	% For LaTeX tables use
	\begin{tabular}{cccccc}
		\hline\noalign{\smallskip}
		$\sqrt{s}$ (TeV) &  1.96  & 13  & 14  & 27 & 100\\
		\noalign{\smallskip}\hline\noalign{\smallskip}
		$\Delta c_{LXX} / \Delta c_{LXY}$ & $1\times 10^{-1}$  & $2\times 10^{-4}$ & $2\times 10^{-5}$ & $4\times 10^{-6}$ & $1\times 10^{-6}$ \\
		$\Delta c_{LXZ} / \Delta c_{LYZ}$ & $8\times 10^{-2}$ & $5\times 10^{-4}$ & $9\times 10^{-5}$ & $2\times 10^{-5}$ & $4\times 10^{-6}$ \\
		\noalign{\smallskip}\hline\noalign{\smallskip}
		$\Delta c_{RXX} / \Delta c_{RXY}$ & $9\times 10^{-2}$  & $4\times 10^{-4}$ & $9\times 10^{-5}$ & $2\times 10^{-5}$ & $5\times 10^{-6}$ \\
		$\Delta c_{RXZ} / \Delta c_{RYZ}$ & $7\times 10^{-2}$ & $2\times 10^{-3}$ & $3\times 10^{-4}$ & $6\times 10^{-5}$ & $2\times 10^{-5}$ \\
		\noalign{\smallskip}\hline\noalign{\smallskip}
		$\Delta d_{XX} / \Delta d_{XY}$ & $1\times 10^{-1}$  & $1\times 10^{-4}$ & $2\times 10^{-5}$ & $3\times 10^{-6}$ & $8\times 10^{-7}$ \\
		$\Delta d_{dXZ} / \Delta d_{YZ}$ & $7\times 10^{-2}$ & $4\times 10^{-4}$ & $7\times 10^{-5}$ & $1\times 10^{-5}$ & $3\times 10^{-6}$ \\
		\noalign{\smallskip}\hline
	\end{tabular}
\end{table}

\end{document}